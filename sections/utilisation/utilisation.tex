\section[Utilisation]{Utilisation\label{sec:utilisation}}

\subsection[Commandes]{commandes}
Avant de créer votre premier projet, nous allons voir les commandes de base pour utiliser \laravelsail{}.

Pour utiliser le script, il vous faudra utiliser la commande suivante dans le dossier du projet (que vous créerez à la section suivante):

\begin{lstlisting}
    ./vendor/bin/sail [command] [options] [arguments]
\end{lstlisting}

Comme taper \verb|./vendor/bin/sail| devant chaque commande est fort fastidieux, vous pouvez créer un alias pour sail en ajoutant la ligne suivante tout en bas de votre fichier \verb|.bashrc| (pour \windows{} (dans WSL2)) ou \verb|.zshrc| pour \macos{}. Ces fichiers devraient se trouver dans les dossiers sources (\verb|~|)

\begin{lstlisting}
    alias sail='[ -f sail ] && sh sail || sh vendor/bin/sail'
\end{lstlisting}

Ce qui vous permettra de ne plus devoir taper \verb|./vendor/bin/sail| à chaque commande mais simplement \verb|sail|.

\subsection[Démarrage du container]{Démarrage du container}
    Pour lancer le container et ainsi pouvoir travailler sur le projet, entrez dans le dossier de votre projet via un terminal et lancez la commande\footnote{Si un container est déjà lancé, vous aurez une erreur! Avec seulement \laravelsail, on ne peut lancer qu'un container à la fois.}:
    
    \begin{lstlisting}
        sail up -d
    \end{lstlisting}

    Cette commande lancera le container en mode détaché pour garder l'accès au terminal (pratique puisque vous devrez exécuter de nombreuses commandes, même une fois le container démarré)\footnote{On peut aussi lancer dans un terminal avec \textit{sail up}}

    C'est une fois cette commande faite (et les autres configurations effectuées) que vous pourrez accéder à votre site en allant sur \url{http://localhost/} ou votre URL personnalisée.

\subsection{Arrêt du container}

Pour éteindre le container lorsque vous avez finis de travailler (mais le laisser déployé dans \dockerdesktop), tapez la commande

\begin{lstlisting}
    sail stop
\end{lstlisting}

Enfin, pour éteindre le container et le supprimer, tapez

\begin{lstlisting}
    sail down
\end{lstlisting}
