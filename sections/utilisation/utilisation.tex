\section[Utilisation]{Utilisation\label{sec:utilisation}}

Pour utiliser le script, il vous faudra utiliser la commande suivante dans le dossier du projet (sail est installé par defaut lors de la création du projet):

\begin{lstlisting}
    ./vendor/bin/sail [command] [options] [arguments]
\end{lstlisting}

Vous pouvez créer un alias pour sail en ajoutant la ligne suivante à votre .bashrc, zshrc, ect: 

\begin{lstlisting}
    alias sail='[ -f sail ] && sh sail || sh vendor/bin/sail'
\end{lstlisting}

\subsection[Démarrage du container]{Démarrage du container}
    Pour lancer le container et ainsi pouvoir travailler sur le projet, entrez dans le dossier de votre projet via un terminal et lancer la commande:
    
    \begin{lstlisting}
        sail up -d
    \end{lstlisting}

    Cette commande lancera le container en mode détacher pour garder l'accès au terminal (pratique si l'on travaille sur VSCode)\\
    \footnotesize{On peut aussi lancer dans un terminal avec \textit{sail up}}


    Vous pouvez maintenant avoir accès à la page de votre projet en allant sur \textit{localhost} via un moteur de recherche comme Mozilla, Google, ect.

\subsection{Arrêt du container}

Pour éteintre le container lorsque vous avez finis de travailler, tapez la commande

\begin{lstlisting}
    sail down
\end{lstlisting}